\documentclass[a4paper,12pt]{report}

\usepackage[utf8x]{inputenc}
\usepackage[english]{babel}
\usepackage[T1]{fontenc}
\usepackage{lmodern}

\usepackage{amsmath}
\usepackage{amssymb}
\usepackage{geometry}
\usepackage{a4wide}
\usepackage{enumerate}
\usepackage{graphicx}
\usepackage{lastpage}

\linespread{1.2}

\usepackage{fancyhdr}
\pagestyle{fancy}
\fancyhead{}
\fancyfoot{}

\newcommand{\myname}{Axel Angel}
\newcommand{\mytitle}{Properties of Convolutional Neural Networks}
\newcommand{\mysubtitle}{Report}
\newcommand{\mydate}{\today}

\lhead{\myname}
\rhead{\mytitle}
\chead{\mysubtitle}
\rfoot{Page \thepage\ of \pageref{LastPage}}
\lfoot{\mydate}

\pdfinfo{
    /Author (\myname)
    /Title (\mytitle)
    /Subject (\mysubtitle)
    }

\begin{document}
\begin{abstract}
    % TODO: rewrite
    Current research in Computer Vision has shown that Convolutional Neural Networks (CNNs) give state-of-the-art performance in many classification tasks and Computer Vision problems.
    The embedding of CNNs, which is the internal representation produced by the last layer, can indirectly learn interesting topological and relational properties.
    By using a suitable loss function, these models can learn invariance to a wide range of non-linear distortions such as rotation, viewpoint angle or lighting condition.
    In this work, we give useful insights about CNNs embeddings and we propose a new loss function, derived from the contrastive loss, whose mapping under particular distortions is predictable.
    Given an input image, only a single forward pass is necessary to generate the outputs of all possible distortions, whereas standard models would require computing all combinations of distortions which is too expensive in practice or invariant models would lose the distortion information.
    % ^ TODO: careful about this last statement, too strong!
\end{abstract}

\chapter{Introduction}
\begin{verbatim}
common use of high-dim data, like images
dim reduction (LLE, isomap, t-SNE), why, drawbacks
an other approach is to use CNN to classify, then dim reduction on embedding
allows to understand what's happening in CNN, better topology
distortions are spread in the embedding
what about learning invariance? but keep information
need mapping (like DrLIM), train with siamese network
predictable mapping << our contribution
need quantitive measures (to compare) << our contribution
new loss function, short summary of results
use case (medical imaging)
\end{verbatim}

\section{Thesis Outline}
% list chapters with their summarzied content

\chapter{Related Work}
% explain each paper, group them and discuss result adv/drawbacks

research found NN are good to classify lots of problems. CNN are naturally more inclined for computer vision (accuracy and speed).

deeper architectures improves accuracy on more complex tasks, people found optimisations to make them fast.

paper: Flexible, high performance convolutional neural networks for image classification

Even though they give impressive results and are widely used in computer vision, we still lack important formal understanding why they work so well.

paper: Intriguing properties of neural networks

CNN are complex and strange results were found. Research found multiple hypothesises.

% adversial attacks, linearity of networks.
% Towards Deep Neural Network Architectures Robust To Adversarial Examples
% Analysis of classifiers' robustness to adversarial perturbations

It is not possible to directly look into the embedding, due to high dim space. however people uses dim reduction to create a human viewable representation of this space.
lots of dim reduction exists

paper: Stochastic Neighbor Embedding

proposes a maximization problem that compute from a high-dim dataset a 2D embedding preserving neighborhood distances. Kullback-leibler divergence.

paper: Visualizing Data using t-SNE
proposes a variation of Stochastic Neighbor Embedding, easier to optimize, produces better 2D map by spreading points more evenly and preserving topological structures

we can see how CNN are clustering samples depending on multiple factors.
people add distorted samples to their dataset (data augmentation) so models can learn to be transfo invariant, but results shows the embedding cluster them into clusters per transfo instead of per label [XXX].
altought it is not intuitive, this can be seen by inspecting the embedding.
we can add constraints against distortions directly over the embedding, to cluster them the way we meant.

paper: Encoded Invariance in Convolutional Neural Networks

encoding invariance in the learnt embedding which preserves the signal. Formal analysis of impact of rigid motions versus deformations on classification. Proposing wavelet networks.

paper: Deep Symmetry Networks

proposes an alternative network, symmetry networks, to extend invariance beyond translation by using feature map over arbitrary symmetry groups by use of kernel-based interpolation.

paper: Dimensionality Reduction by Learning an Invariant Mapping

proposes to creates a dimensionality reducing mapping by training a CNN using a new loss function, the contrastive loss, based on energy models. The results on NORB is a mapping from image space to a 3D cylinder where two axes describes the 2D orientation and the third axe describes the azimuth angle, the mapping is invariant to illumination. Results on 4s and 9s of MNIST shows that CNN can learn a mapping that separates labels and make similar digits close despite translations.

\chapter{Theory: Dimensionality reduction}

\section{Optimisation Problem}
optimisation problem that optimize visualization in 2D like t-SNE

introduce theory of t-sne and such

\section{CNN Classifier}
introduce CNN and embeddings

we can use them as classifier and take the last layer as a smart dimension reduction (better than PCA), then use t-sne

\section{CNN for Reduction}
introduce contrastive loss

we can go further and directly ask the CNN to learn a mapping that ressembles t-sne.

\section{CNN for Predictable Reduction}
introduce our loss function

instead of being invariant, we can optimize the embedding to be predictable in certain way, for eg: quantifying one distortion per axis.

\chapter{Methodology}

explain our models

use of Caffe

introduce datasets: mnist and norb.

how we create our train/test dataset

how are models trained in caffe

how we can compute our energy-distance for comparisons.

\chapter{Results and Dicussion}

tables and graphs: loss of training (for major models), add accuracy for comparison, compare lecun with ours (energy-distance).

\section{t-SNE on LeNet}
explain what we found using t-sne on last layer of lenet, on mnist.

\section{Contrastive LeNet}
explain what we found using double-contrastive loss with lenet, on mnist.

can add norb, when finished.

\chapter{Conclusion}

% bibliography

Caffe: Convolutional Architecture for Fast Feature Embedding
proposes a deep-learning framework focusing on CNN, easy experiment, easily modifiable source, reference models (lenet, imagenet), scalable, GPU, active community.

DIY Deep Learning for Vision: a Hands-On Tutorial with Caffe
caffe tutorial

% required by NORB dataset
Learning Methods for Generic Object Recognition with Invariance to Pose and Lighting. IEEE Computer Society Conference on Computer Vision and Pattern Recognition (CVPR) 2004
\end{document}
