\documentclass[a4paper,12pt]{report}

\usepackage[utf8x]{inputenc}
\usepackage[english]{babel}
\usepackage[T1]{fontenc}
\usepackage{lmodern}

\usepackage{amsmath}
\usepackage{amssymb}
\usepackage{geometry}
\usepackage{a4wide}
\usepackage{enumerate}
\usepackage{graphicx}
\usepackage{lastpage}
\usepackage{cite}
\usepackage{hyperref}

\linespread{1.2}

\usepackage{fancyhdr}
\pagestyle{fancy}
\fancyhead{}
\fancyfoot{}

\newcommand{\myname}{Axel Angel}
\newcommand{\mytitle}{Towards Predictable Embedding of Convolutional Neural Networks}
\newcommand{\mysubtitle}{Report}
\newcommand{\mydate}{\today}

\newcommand{\N}{\mathbb{N}}
\newcommand{\Z}{\mathbb{Z}}
\newcommand{\Q}{\mathbb{Q}}
\newcommand{\R}{\mathbb{R}}
\newcommand{\C}{\mathbb{C}}

\lhead{\myname}
\rhead{\mytitle{} - \mysubtitle{}}
\chead{}
\rfoot{Page \thepage\ of \pageref{LastPage}}
\lfoot{\mydate}

\pdfinfo{
    /Author (\myname)
    /Title (\mytitle)
    /Subject (\mysubtitle)
    }

\newcommand{\eg}{e.g.}


\begin{document}
\thispagestyle{empty}
% TODO: make
First page (make guard, epfl logo, prof names, title, etc)

\newpage
\begin{abstract}
    % TODO: rewrite
    Current research in Computer Vision has shown that Convolutional Neural Networks (CNNs) give state-of-the-art performance in many classification tasks and Computer Vision problems.
    The embedding of CNNs, which is the internal representation produced by the last layer, can indirectly learn interesting topological and relational properties.
    By using a suitable loss function, these models can learn invariance to a wide range of non-linear distortions such as rotation, viewpoint angle or lighting condition.
    In this work, we give useful insights about CNNs embeddings and we propose a new loss function, derived from the contrastive loss, whose mapping under particular distortions is more predictable.
    Given an input image, only a single forward pass is necessary to generate the outputs of all possible distortions, whereas standard models would require computing all combinations of distortions which is too expensive in practice or invariant models would lose the distortion information.
    % ^ TODO: careful about this last statement, too strong!
    Moreover we introduce a simple method to fairly compare quantitatively embeddings whose purpose is to capture topological structures of particular distortions.
\end{abstract}

% ==============================================================================
\tableofcontents


% ==============================================================================
\chapter{Introduction}
common use of high-dim data, like images

dim reduction (LLE, isomap, t-SNE), why, drawbacks

an other approach is to use CNN to classify, then dim reduction on embedding

allows to understand what's happening in CNN, better topology

distortions are spread in the embedding

what about learning invariance? but keep information

need mapping (like DrLIM), train with siamese network

predictable mapping << our contribution

need quantitative measures (to compare) << our contribution

new loss function, short summary of results

use case (medical imaging)

\section{Thesis Outline}
% list chapters with their summarzied content


% ==============================================================================
\chapter{Related Work}
% explain each paper, group them and discuss result adv/drawbacks

More than 20 years ago, researchers discovered that {\em Neural Networks} (NN) is a highly flexible model architecture to solve many classification problems.
Today, many state-of-the-art models use improved NN for computer vision tasks like MNIST \cite{mnist_web} and ImageNet \cite{krizhevsky2012imagenet}, face detection as well \cite{rowley1998neural} and many more \cite{prechelt1994proben1}.

Most of today models are using {\em Convolutional Neural Networks} (CNN) which are a specialized variant working well for visual tasks.
This architecture is more efficient for computer vision problems, partly because the convolutions share their weights in the first layers to extract spatial cues.

Current researches is mainly focused on advancing into more complex classification problems using Deep Learning.
This field suggests to improve the current results by deepening the architecture in terms of number of layers to express more abstract and higher-level concepts.
Unfortunately, more layers require more computations than before but researchers started to overcome this challenge.
Ways to speed up the training and the classification appeared and were greatly beneficial for the development of this field in recent years\cite{ciresan2011flexible}.
%paper: Flexible, high performance convolutional neural networks for image classification
Both researchers and professional programmers built various frameworks to train and to use NN with many different targets, like: performance, accessibility or composability.
The most popular are: Theano\cite{bastien2012theano}, Caffe\cite{jia2014caffe}, Torch7\cite{collobert2011torch7}, Pylearn2\cite{goodfellow2013pylearn2}.
We decided to leverage the efficiency and modularity proposed by Caffe for several reasons described later.
%deeper architectures improves accuracy on more complex tasks, people found optimisations to make them fast.

Even though CNN models are now widely used in computer vision with great success thanks to these frameworks, current research exposed our ignorance of these networks behaviors.
Some papers work on reliable ways to fool networks using adverserial attacks which exploits their unintuitive properties\cite{szegedy2013intriguing}.
Most publications don't try to formally justify their good results because it just works. % FIXME: citation needed?
%paper: Intriguing properties of neural networks
Indeed, relations between units are complex due do their non-linear nature, this problem is stressed by distortions present in the inputs as well.
Moreover the input-output relation, when inputs are kept distorted on purpose, is usually either killing the signal (invariance) or not in a predictable manner (controlled preservation).

%CNN are complex and strange results were found.
%Research found multiple hypothesises.

% adversial attacks, linearity of networks.
% Towards Deep Neural Network Architectures Robust To Adversarial Examples
% Analysis of classifiers' robustness to adversarial perturbations

The manifestation of this relation can be looked visually using methods to process the output into an other form.
It is not possible to directly look at the high-dimensional embedding, because we cannot easily plot nor interpret so many dimensions.
Dimensionality reduction methods is a popular tool in such case and several methods exist that are specialized for visualization.

% moved from dim red chapter
Standard methods like PCA, MDS for example, are used for reducing dimensions usually as a preprocessing step and they are limited to linearities of the data.
Their primary characteristic is to maximize the variance of the original data which is important for reconstruction but not necessarily for visualization.
Other non-linear dimensionality reduction methods, use an optimisation algorithm with a loss function which improve their flexibility for other purpose like visualization.
They preserve local structure/cluster like: isomap, LLE, SNE, (and more).
Most of the methods fails to keep global clusters and local details at the same time where as the latest papers show SNE a better potential\cite{SNE}.
%paper: Stochastic Neighbor Embedding

SNE has introduced a better maximization problem to preserve point neighborhood and general clusters with an important emphasis on distances\cite{SNE}.
%Kullback-leibler divergence.
However it suffers from ``center-crowdedness'' and has difficulties in optimisation\cite{t-SNE}.
That's why papers are using t-SNE, a variant of SNE, to reduce CNN embeddings into a 2D human-friendly manifold. % FIXME: more cite?
This method exhibits an easier optimisation formulation and it preserve more structures at diverse scales.
More convincing examples were created with t-SNE directly from raw popular datasets like MNIST\cite{t-SNE} and more\cite{van2009new}.

It has become possible to easily see many more details in CNN embeddings: how samples are grouped depending on multiple factors: essentially based on similarity of the digits (strokes, thickness and shapes) and natural variance (rotation).
%Although it is unintuitive, this can be seen by inspecting the embedding.
%we can add constraints against distortions directly over the embedding, to cluster them the way we meant.
Many papers use dimensionality reduction to create a human viewable representation of the output embedding \cite{donahue2013decaf}\cite{yu2014visualizing}\cite{yaotiny}.
Only a few papers actually try to formalize the input-output relations\cite{goodfellow2009measuring}.
Some papers proposes to learn transformation-invariant embeddings, by means of modelling distortions directly into their models\cite{gens2014deep} or using data-augmentation\cite{hadsell2006dimensionality}.
%paper: Encoded Invariance in Convolutional Neural Networks
%paper: Deep Symmetry Networks
%proposes an alternative network, symmetry networks, to extend invariance beyond translation by using feature map over arbitrary symmetry groups by use of kernel-based interpolation.
%paper: Dimensionality Reduction by Learning an Invariant Mapping

The latter proposes to creates a dimensionality reducing mapping by training a CNN using a new loss function based on energy models, called the {\em contrastive loss}.
Their results using MINST shows that CNN can learn a mapping that distinguish labels, and group alike digits even when they are translated artificially.
They experimented with NORB as well with intriguing results: each sample is mapped on a 3D cylinder whose axises quantify the orientation in 2D and the azimuth angle in 1D.
The network was successfully trained to ignore lighting conditions, which is a strong non-linear distortions.
They use the Siamese training architecture to present image pairs which are optimized to be close if similar or far otherwise\cite{bromley1993signature}\cite{chopra2005learning}.
There are several advantages of such method: the cost penalty is very low and only present in training, the usage of standard CNN allow more freedom for experiments with different architectures and the proposed loss function is effective and simple.

However distortions are damped instead of being quantified in a predictable way where we think the distortion information is valuable later on.
Moreover they describe qualitatively the coherence of the embedding but doesn't provide nor propose a formal way to measure and compare its quality.
We used this paper as our primary reference for experiments to continue improving their ideas in this work.


% ==============================================================================
\chapter{Theory: Neural Networks}
We introduce the general layout of neural networks and their mathematical foundations.
The important differences between NN and CNN is discussed and we briefly justify why CNN is so predominant in computer vision tasks.
An intuitive explanation of embeddings is given and will be used later for dimensionality reduction.

\section{NN and CNN Classifier}
%introduce CNN and embeddings

% FIXME: add figures and formulas
nn and cnn are composed of inter-connected layers of units, or neurons.
to classify an input we fed the first layer of the network.
one layer does a single particular computation over its input, usually by adding a bias then followed by a non-linear function, which is then propagated through its connections to the next layer.
the feedforwarding continues until we reach the last layer which is the output of the network.
usual nn are characterized by: neuron computation is a simple inner product between its input and its own weight followed by a sigmoid; each unit is fully connected to all units in the next layer.
training the weights of such network generally involve gradient descent which minimizes the loss function of the network.
This loss function gives the error between the prediction of the network, its output versus the expected output, which is in classification the class label.
To train, the input is first forwarded into the network to compute the error, then this latter is back-propagated up to the first layer, while each unit weights is adjusted to minimize the unit error.
This process should be repeated until the error on the validation set has converged to a local optimum.

cnn is a special case of nn by adding certain restrictions, working well experimentally with computer vision problems such as image classification.
Three main ideas differentiate them: local receptive fields, weight sharing and subsampling.
in cnn, a convolutional layer can model receptive fields by computing multiple trainable 2D kernels which is convoluted with its entire input whose result is called its feature maps.
We can see a kernel convolution as the replication of a single unit along the dimensions of the input (a 2D grid for images), thus all weights are constrained to be equal by definition.
a convolutional layer contains multiple units where each has its own kernel, thus there are different kernels applied to the image, where a single kernel convolution is called a feature map.
Such layer naturally computes filters which can be seen like feature extractors driven by data.
the benefit of weight sharing in convolutional layers is to reduce the number of global parameters to improve generalization.
usual cnn add a subsampling layer right after a convolutional layer to reduce the dimensionality of the feature maps.
moreover most cnn architectures connects convolutional-subsampling layers to the input which extract features, which is then feed into a regular nn.
Thus we can see a cnn as two parts: a trainable feature extractor which is fed into a processing nn.
meanwhile cnn have more constrained, it has been shown they generally outperforms nn with more invariance in multiple computer vision problems \cite{simard2003best}\cite{mnist_web}\cite{lawrence1997face}\cite{krizhevsky2012imagenet}.

as previously said, researches have shown with t-sne that nn trained for classification have interesting structures in their last-layer embedding.
the choice of the layer is justified by the fact that: deeper layers extract more high-level informations, which is necessary to separate classes, moreover the prediction is a direct product with the last layer, which encourage the network to have a simple structure directly clustered by classes.
that's why this last layer tend to cluster samples of the same class together while separating the rest\cite{donahue2013decaf}\cite{yu2014visualizing}.
thus it is reasonable to evaluate the quality of this embedding as a dimensionality reduction method with great potential to capture more information compared to PCA which captures linear correlations.
it is important to note that t-SNE plays an important role in the quality of this 2D embedding by preserving structures but it should be said also that t-sne cannot separate the original dataset as well as with the help of the classifier features.
Because t-SNE has no knowledge of the labels, it is the classifier who improves the quality of the structures of the final embedding as we'll show later, because it learned specialized filters to distinguish the classes.
This justifies why we combined CNN classifiers with t-SNE to create a 2D embedding for our visualisations.


% ==============================================================================
\chapter{Theory: Dimensionality reduction}
Dimensionality reduction is an important method in machine learning that maps points in a high-dimensional space into a space with a lower number of dimensions.
We will call this lower subspace an {\em embedding}.
Representations suitable for human visualization should keep important relations between points of the original space.
In our experiments, we will apply dimensionality reduction especially on our neural networks' outputs.
This allows us to quickly compare (qualitatively) the impact of controlled distortions applied on images over networks' outputs.
The structure inferred by distances between points such as clusters, intra-cluster neighbors and inter-cluster outsiders reveals important properties of neural network models.
Thus we need accurate low-dimensional representations of embeddings that preserve local distances to compare different models.
In the following sections, we will discuss dimensionality reduction methods that weren't fit for our uses, then we introduce the one we used, called t-SNE, in details.

\section{Optimisation Problem}
Various algorithms differentiate themselves by several properties: their goal (\eg: interpolation, compression, visualization), by what they preserve (\eg: variance, distances), how they model point correlation (\eg: linearly or not) or whether they can model new points (\eg: a representation or a mapping).

First, our choice is guided by our primary need, which is to visualize our data.
Thus the method should produce a 2D or 3D embedding, preferably in 2D for scatter plots easily interpretable.
Secondly, we need a guarantee on the preservation of relations between points, especially the distances keeping them clustered or not.
Considering the previous works we discussed above, t-SNE is the best candidate in the current state-of-the-art.
We now define the method objective functions, probabilities and explain intuitive properties.

%introduce theory of SNE, then t-SNE modification.
Let's define input space $X = \left\{ x_1, x_2, \dots, x_n \right\}$ then the goal is to map into a lower-space $Y = \left\{ y_1, y_2, \dots, y_m \right\}$, where $n$ is very high (around a few hundreds to thousands, \eg: mnist pixels dimensions: 784) and $m$ is $2$ or $3$.
SNE uses two Gaussian distributions for each point expressing the neighbor distances in $X$ and its equivalent in $Y$.
The Kullback-Leibler divergence is used to compute the objective function, which represents the mismatch of these two distribution for each pair:
\begin{eqnarray}
    L = \sum_i KL(P_i || Q_i) = \sum_i \sum_j p_{j|i} \log\left(\frac{p_{j|i}}{q_{j|i}}\right)
\end{eqnarray}
The probability for a pair of point $x_i$ electing $x_j$ in $X$ follows a Gaussian is as follow:
\begin{eqnarray}
    p_{j|i} = \frac{\exp(-|x_i - x_j|^2 / 2 \sigma_i^2)}{\sum_{k \not = i} \exp(-|x_i - x_k|^2 / 2 \sigma_i^2 )}
    %&
    %q_{j|i} = \frac{\exp(-|y_i - y_j|^2)}{\sum_{k \not = i} \exp(-|y_i - y_k|^2)}
\end{eqnarray}
The equivalent probability $q_{j|i}$ in $Y$ is the same except that $\sigma = 0$ and $x$ is replaced by $y$.
Therefore, a closer pair implies a higher neighbor-election probability because the distance is low.
This cost function gives an asymmetrical importance to the distances: nearby points in $X$ are greatly penalized if they are far in $Y$; whereas a small cost is incurred for pairs far in $X$ but close in $Y$.
%There is a simple physical interpretation of the SNE optimisation problem: pairs are modeled by asymmetrical springs in a mechanical system, the best solution is when the system is still.

As we said SNE suffers from crowdedness problems in the middle of the embedding and the optimisation is harder due to the asymmetrical nature of the objective function.
Both problems were addressed in t-SNE which give very good results in practice.
The two major differences with t-SNE is the symmetrization of the cost function and replaces the distributions of the embedding by student variants.
In t-SNE, a single objective function is minimized:
\begin{eqnarray}
    L = \sum_i KL(P || Q) = \sum_i \sum_j p_{ij} \log\left(\frac{p_{ij}}{q_{ij}}\right)
\end{eqnarray}
where:
\begin{eqnarray}
    p_{ij} = \frac{p_{j|i} + p_{i|j}}{2 n}
\end{eqnarray}
which forces outliers points to contribute more to the loss.
And:
\begin{eqnarray}
    q_{ij} = \frac{1 / (1 + |y_i - y_j|^2)}{\sum_{k \not = l} 1/(1 + |y_k - y_l|^2)}
\end{eqnarray}
which replaces the $q$ distribution in SNE by a Student with a heavier tail: distances in high dimensional spaces spread across more dimensions; in low dimensional space, the accurate equivalent distance needs to be much higher per dimension (thus more points end up farer in $Y$, a heavier tail than in $X$).
As before, a point cannot elect itself: $p_{ii} = q_{ii} = 0$ and probabilities are symmetric for both distributions: $p_{ij} = p_{ji}$ and $q_{ij} = q_{ji}$.
As stated previously the input space $X$ has much more dimensions were distances can be expressed than the 2 dimensions of $Y$.
In summary, t-SNE uses Kullback-Leibler divergence to minimize the mismatch of these two spaces by means of probabilities, therefore the chance of important local structures (frequent patterns) being preserved is higher than with other methods (mosts don't express this goal through an objective function).
Global structures is encouraged by the coherence of the local structures as the divergence decreases and the system stabilizes.

\section{CNN for Dimensionality Reduction}
Usual dimensionality reductions like t-SNE are helpful for many visualizing tasks but it has important drawbacks as well.
We think the most important ones are: the computational cost, the incapacity of mapping new points and the indirect control over the resulting embedding.
Current implementations of t-SNE are still rare, unpolished and require tricks to make them tractable in practice (\eg: reducing first with PCA).
Fortunately there are new promising alternatives emerging directly harnessing the power of NN.
We introduce such models in the following section.

%there is a direct way to learn an embedding with CNN without t-SNE.
CNN are mostly used for classification but we can optimize them for other purposes as well.
Instead of the softmax loss function used for classification, we can use a special training architecture with a suitable loss that when optimized tries to satisfy some constraints.
For example to create an embedding with chosen properties directly found into the output layer.
Moreover the ideas presented by reduction methods like t-SNE can be formulated in different terms to be applicable to NN.
In our case the most important idea is to keep similar points together and dissimilar far.

The Siamese network combined with a contrastive loss is a good practical solution to train such networks.
Let us define $G_W$, the function that computes the network output, with parameters $W$.
Then the Siamese network put two weight-sharing instances of $G_W$ side by side, each having their own separate input.
This architectural idea is to create two instances of $G_W$ where their weights $W$ are shared.
On top of this Siamese network is placed the ``cost module'', the contrastive loss, which will compute a loss proportional to the difference between the two output.
The complete network takes a pair of images as input, each image fed into a single $G_W$ instance, and the output is computed over their outputs.
When the network is used, only a single instance $G_W$ is required to feed an input which gives the dimensionality-reduced point.
% FIXME: figure.

The contrastive loss function is based on energy models but a single attractive term is not sufficient because it would allow degenerate solutions where all points squashed together; thus an opposing term to push dissimilar pairs should appear as well.
The definition of the contrastive loss for a pair is as follow:
\begin{eqnarray}
    L = \frac{1}{2} Y (D_W)^2 + \frac{1}{2} (1-Y) \max(0, m - D_W)^2
\end{eqnarray}
where $Y \in \{0,1\}$ is the label: $1$ for similar pairs, $0$ otherwise; $D_W \in \R^N$ is the difference between the two network outputs and the parameter $m \in \R$ defines the minimal distance between dissimilar points.

With this cost function, a label describes whether or not a pair should be close in the embedding.
The definition of similarity is left open for the application at hand.

\section{CNN for Predictable Reduction}
The method above can learn an embedding which tries to pair the way it was intended, but there is still room for improvement.
First, the structure of the resulting embedding is determined by the dataset and again we indirectly shape it by giving pairing information which helps to constraint local structures.
However, there is no guarantee that this system will converge to an intended global structure, for example having certain deformations in a predictable ordering (\eg: from lowest to highest) or shape (\eg: a line, plane or circle).
Secondly, the training required to make the embedding may converge to a desirable structure after a long time or never even with multiple run using different seeds due to the lack of constraint.
The number of embedding dimensions, $n$, is higher than the dimension of the pairing, $1$, which is why such model is not predictable, because these dimensions gives more freedom to the model in the way to represent these pairs arbitrarily.

In the intend to improve this solution, a possibility is to add more constraints directly into the optimisation process.
We propose to give more informations for each pair which can be used in the loss function to create shape and ordering enforcements.
This method allows to control separately the usage of each dimension of the embedding to express different properties of our dataset.
To achieve this goal, we generalize the contrastive loss to pairs having $n$ pairing labels represented into an $N$-dimensional embedding, where $n \leq N$ by definition.
The relation between a pairing label and its binds to certain dimensions, should be defined per a use-case basis, as the pair relation.

We now introduce our loss function with a formal definition.
Let's define $N$ as the number of embedding dimensions, $n$ the dimensions of the labels, where $n \leq N$ as before.
Let's define $D \in \N^n$ as the number of embedding dimension assigned per label, then we restrain ourself to: $\sum_i^n D_i = N$, which states each dimension should be assigned to a single label.
The definition of the generalized $n$-dimensional contrastive loss for a pair is as follow:
\begin{eqnarray}
    L = \frac{1}{2} \sum_{i=0}^n \left( Y_i (D_{Wi})^2 + (1-Y_i) \max(0, m_i - D_{Wi})^2 \right)
\end{eqnarray}
where $Y \in \{0,1\}^n$ with $Y_i$ being the $i$th component of $Y$, $D_{Wi} \in \R^{D_i}$ is the difference between the two points in the sub-embedding for dimensions of the $i$th component, and $m_i \in \R$ is the minimal distance for dimension $i$.


% ==============================================================================
\chapter{Methodology and Results}

In this chapter, we will describe the general environment we used to produce our experiments, the settings to reproduce them and their results.
The different models we used are first introduced in details with their architecture.
Followed by the explanation of our implementation using Caffe and Python.
We quickly discuss how our datasets are generated based on MNIST and NORB.
We present what we use to quantitatively measure our results that we use later for comparisons with prior art.

\section{Models and parameters}

%explain our models.
As previously said, we are closely following the work of our reference paper \cite{hadsell2006dimensionality}.
Therefore, We employed the two different models they choose to experiment with the two popular datasets MNIST and NORB, one model per dataset.
The first model is {\bf LeNet 5}, a multi-layer neural networks that features an architecture specialized for handwritten characters.
It is with no surprise that they are using it for MNIST, which is a handwritten digit dataset.
We are using a very close fine-tuned variant with minor changes.
This network architecture is comprised of a total of 7 layers with trainable parameters.

The first part of this network is comprised of two pairs of convolution-pooling layers, where convolution kernels share their weights.
As explained earlier, each convolution is applied to the entire image to create a single feature map.
The first convolutional layer has 20 different trainable $5 \times 5$ kernels, which is followed by a trainable $2 \times 2$ max-pooling layer (with a stride of $2$).
The second convolutional layer has 50 trainable $5 \times 5$ kernels, followed again by a $2 \times 2$ max-pool layer (stride $2$).
The convolutional network ({\em convnet}) part, made by these two pairs of layers, are {\em feature extractors} for this network, it will extract features related to high-level cues to detect digits (\eg: straight or curved strokes).

The second part of this network is composed of two fully-connected layers where only the first one is followed by a non-linear layer.
The first layer has 500 trainable units computing an inner-product, followed by a non-trainable Rectified Linear Unit (ReLU) activations\cite{nair2010rectified}, followed by 10 trainable output units computing an inner-product which gives the digit class (1 versus rest).

The second model is composed of only two fully-connected inner-product layers.
This network is much simpler than the one above because it is trained on a subset of the NORB dataset, which exhibits very few variability (compared to MNIST).
The two layers are made of 20 and 3 trainable units respectively, without non-linearity between them. % FIXME: non-linearity, really?

In the following experiments, we will train our variant of LeNet in two different ways: for digit classification to analyze the ``natural'' embedding and with a Siamese training architecture with the contrastive loss function to analyze a ``constrained'' embedding.
For our second model, we will only train a ``constrained'' embedding with the Siamese architecture for NORB, like in our reference paper.
It is important to note that in the LeNet ``constrained'' experiment we keep only a single fully-connected layer with the number of trainable units equals to the embedding dimensions (\eg: $2$ or $3$).
The motivation is that the classification task is built on top of the network stack generating this embedding but in our second experiment we need this output rawly, like in our reference paper.

In the case of digit classification, the model is trained using Stochastic Gradient Descend (SGD) with a learning rate of 0.01 optimizing a SoftMax loss function.
Siamese models are trained with SGD with a learning rate between 0.01 and 0.001 optimizing the contrastive loss.

For this project we chose the {\bf Caffe} deep-learning framework to train our models.
There are several reasons among them: the simplicity to express networks, train or manipulate them; Caffe includes several network implementations for MNIST with LeNet and Siamese networks, for ImageNet with AlexNet, GoogLeNet where pre-trained versions are available; Moreover Caffe is well optimized C++ with CUDA GPU parallelism, it is flexible including official Python and MatLab bindings and its community is very active and helpful.

For our experiments on MNIST we could easily adapt the provided LeNet to match the architecture design discussed above.
For the second non-convolutional network, we wrote the network definition ourself due to the triviality of this task.
The training stage need many special parameters related to the solver (SGD): learning rate, batch size, momentum, weight decay (gamma and power) and number of epoch.
They were kept unchanged because the Caffe community already fine-tuned them.
We limited all our experiments to $10'000$ iterations, with a batch size of $64$.
All our networks were trained using the Caffe built-in solver provided with the binary using {\tt caffe train}.

% FIXME: add figures

Most of our experiments were implemented using the Python scripting language: to generate our transformed datasets, to compute quantitative measures and visualization projections.
We created our own set of tools to generate distorted training sets for our t-SNE experiments and to generate the distorted pairs training sets for our two Siamese experiments.
The scikit-learn Python library ({\em sklearn}) now implements most of the standard algorithms for machine learning including the one we used: PCA and t-SNE\cite{pedregosa2011scikit}.
The performance of t-SNE depends heavily on the number of input dimensions and in our experiments we had to apply PCA on our datasets first like other papers suggest\cite{t-SNE} (to compute t-SNE in a reasonable amount of time).
We followed sklearn suggesting to always reduce to $50$ dimensions with PCA before applying t-SNE\footnote{t-SNE documentation page: \url{http://scikit-learn.org/stable/modules/generated/sklearn.manifold.TSNE.html}}.
% FIXME: make source available?

use python to feed-forward, test.

For the visualization of our resulting embeddings, we also implemented our own solution based on the web library {\em CanvasJS}.
We use it to make an interactive scatter plots directly from the output of t-SNE or the neural networks.
Several features were already provided: coloring points, fast plotting for interactivity; But some were implemented ourself: zooming and moving the viewport, display the image of any sample and filtering capabilities to our own application.

display sample images, filtering/highlights capabilities, zoom, move.

good insight and feedback of results.

\subsection{Datasets}
introduce datasets: mnist and norb.

mnist use lenet archi.

how we create our train/test dataset

python script, predictable.

pairing method.

reproduce lecun mnist and norb.

we use 3d for mnist, compare with lecun project 2d.

\section{Evaluation Metrics}
contribution: how we can compute our energy-distance for comparisons.

\section{Results and Discussion}

tables and graphs: loss of training (for major models), add accuracy for comparison, compare lecun with ours (energy-distance).

\subsection{t-SNE on LeNet}
show t-sne on mnist
compare to t-sne on classifier embedding.

explain what we found using t-SNE on last layer of lenet, on mnist.

people add distorted samples to their dataset (data augmentation) so models can learn to be transfo invariant, but results shows the embedding cluster them into clusters per transfo instead of per label [XXX].

\section{Contrastive LeNet}
explain what we found using double-contrastive loss with lenet, on mnist.

can add norb, when finished.

usecase of predictable disto


% ==============================================================================
\chapter{Conclusion}

summarize work.

future works.

% bibliography
\bibliography{thesis}{}
\bibliographystyle{plain}

% required by NORB dataset
\nocite{lecun2004learning}

\end{document}
